\section{Transplanting code}
\label{sec:transplating-code:related}

Our proposal for identifying the change that introduced a bug is based on transplanting a test present in the snapshot that fixed a bug to earlier snapshots of the same code. Techniques for transplanting pieces of code were applied by ReDeBug~\cite{jang2012redebug} for fixing clones of a fixed bug, by transplanting the fixing patch. In this respect, the \emph{plastic surgery hypothesis}~\cite{harman2010:automated_patching}, which assumes that changes to source code can be constructed as combinations of other changes already present in the code (grafts) was studied in detail by Barr et al.~\cite{barr2014plastic}, showing how in fact patches could in many cases be transplanted to other areas of code. However, further studies~\cite{castelluccio2019empirical} show how this practice can lead to regressions in the version where the fix is applied.

To determine which specific piece of code of a change in a bug fix is the part actually fixing the bug, BugBuilder~\cite{jiang2021extracting} transplants each part to the previous versions of the code. It then runs the regression test to check whether the bug is still present. This approach is similar to ours, with the difference that the authors of BugBuilder intend to find out which part of a change is the fix, while we intend to find when the bug was introduced, and for that we need to transplant not only to the previous version, but also to many other versions prior to the bug fix.

\newpage

\section{Bug Introduction Changes}
\label{sec:bic:related}

The problem of detecting the change that introduced a bug (BIC) given the change that fixed it (bug fixing change, BFC) has been extensively studied~\cite{sinha2010buginnings,davies2014comparing,sliwerski2005changes,kim2006automatic,williams2008szz,kamei2012large,tantithamthavorn2013mining}. The usual approach has been to assume that the change that introduced the bug touched the same lines of source code that were touched to fix it. The SZZ algorithm~\cite{sliwerski2005changes} was developed based on this assumption. Using the source control management system, it identifies the lines that were edited in the change that fixed a commit, and then which previous changes modified the same lines before the bug was reported. Many variants of SZZ have been proposed, improving the algorithm in different ways. \gema~\cite{rodriguez2018reproducibility} surveyed 187 studies related to SZZ, finding that 38\% of them used the original algorithm, while the rest used a derivative. It also found a very low reproducibility for the studies, with many of them not publishing the implementation of the algorithms they used. Fortunately, this situation is changing, and some years later we have several implementations of SZZ derivatives published~\cite{borg2019szz,lenarduzzi2020openszz,pokropinski2022szz,rosa2021evaluating}.

Another tool (already mentioned) to consider for the same purpose is GitBisect, which through a binary search, assists the developer to locate the commit that introduced the bug. This tool explores the Git history of a project, asking the developer if the bug is in the current commit or not. 
It is therefore up to the developer to perform all the build and testing steps manually. 
An automated bisection over git bisect was proposed, called ``\textit{git bisect run}''~\footnote{\url{https://lwn.net/Articles/317154/}}, which allows the developer to add a script or command to be executed at each step of the tool.
As we will see later, our approach proposes a fully automatic process that allows to obtain much more detailed results of each phase in each commit. 
Our approach also solves a limitation of GitBisect: this tool does not consider the graph structure that the commit history of a project may have, while our approach navigates through the graph with an appropriate algorithm (depth-first search).
Some improvements on this technique have been proposed~\cite{saha2017selective} by selecting commits and tests to save computational costs.

There are studies that reduce the cost for both bisection and SZZ-like blame models such as~\cite{an2021reducing}, which filters commits using the coverage of regression tests for the bug, thus reducing the search space for automated bisection.

The algorithm Delta Debugging of Andreas Zeller~\cite{zeller2002simplifying,zeller2002isolating} use testing to simplify and isolate the failure in the execution trace of some failing test case.
Based on the idea of delta debugging and for the specific case of regression bugs, there are also some techniques that have been proposed. 
For example, the difference between the last version that worked well and the current version where the bug is present can be used~\cite{saha2017selective}, or a combination of the information in the issue tracking system and source code management~\cite{khattar2015sarathi}.

Recently, the performance of SZZ has been studied~\cite{bludau2022pr,petrulio2022szz} in projects that follow the pull-based development
model proposed by GitHub~\cite{gousios2014exploratory}, showing that in this type of projects it is necessary to consider sets of commits when detecting the change that introduced a bug.

None of these techniques deal with automatically transplanting tests to past versions of the source code. 
%\as{Would you not argue that git-bisect does some form of manual transplantation?}
Git bisect and its derivatives do not directly address transplanting code into the past and is limited to a binary search that considers only a linear history model.
SZZ and derived techniques try to infer which changes could have introduced the bug by analyzing the history of the source code.

Our approach, as we will see in detail in Chapter~\ref{chapter:bug-hunter}, aims to go further by running regression tests on the change history of the project.