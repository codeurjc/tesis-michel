\begin{tiny}
    \begin{algorithm}[t!]
        \caption{Algorithm to detect bug-inducing commits}\label{alg:bic}
        \SetKwInOut{Input}{Input}
        \SetKwInOut{Output}{Output}
        \SetKwInOut{Precondition}{Precondition~}
        \Input{A $\mbox{\sl graph}$ (commit history) and the $\mbox{\sl bfc}$}
        \Output{A list of candidates for BIC}
        \Precondition {The $bfc$ status (the output of the regression test) must be success
        % \as{I understand the intention but ``status'' does not seem to be defined. }
        % \michel{I clarify what ``status'' is at text}
        % \as{It might be a good idea to explain ``status'' here such that the algorithm becomes self-contained?}
        }

        $candidates \gets []$\;
        $queue \gets [bfc]$\;
        $visited \gets [bfc]$\;
        $temp\_candidates \gets []$\;
        
        \While{$ NotEmpty(queue) $}{
            $n \gets GetFirst(queue)$\;
            \If{hasFailStatus(n)}{
                $candidates \gets []$\;
            }
            $parents \gets GetParents(graph, n)$\;
            $successParents \gets True$\ \Comment*[r]{Control if all parents are success} 
            % \as{What does $successParents$ mean?}
            % \michel{I clarify at text}
            \If{$ isEmpty(parents) $\Comment*[r]{Reach first commit}}{
                    \eIf{size(queue) == 0}{
                        $break$\;
                    }{
                        $continue$\;
                    }
            }

            \For{$p \in parents$}{
                $successParents \gets successParents \And hasSuccessStatus(p)$\;
                \If{$ \neg hasSuccessStatus(p) $}{
                    \If{$ hasErrorStatus(p) $}{
                        $AddItem(candidates, n)$\;
                    }
                    \If{$ p \notin visited $}{
                        $AddItem(queue, p)$\;
                        $AddItem(visited, p)$\;
                    }
                }
            }
            \If{$ successParents \And \neg hasSuccessStatus(n) $}{
                \eIf(){$ hasFailStatus(n) $}{
                    \Return{$[n]$}\; 
                }{
                    $AddItem(candidates, n)$\;
                    
                    \eIf(){IsEmpty(queue)}{
                        \Return{$candidates$}\;
                    }{
                        $temp\_candidates \gets candidates$\;
                    }
                    
                }
            }
        }
        \Return{$temp\_candidates$} 
    \end{algorithm}
\end{tiny}