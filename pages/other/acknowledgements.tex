\chapter{Acknowledgements}
\specialHead{Acknowledgements}

\begin{FraseCelebre}
    \begin{Frase}
    Journey before destination
    \end{Frase}
    \begin{Fuente}
    Immortal Words, Brandon Sanderson
    \end{Fuente}
\end{FraseCelebre}

Since this section is especially dedicated to people from my close environment who may not be able to deal with English, from the following paragraph I will use the language of Miguel de Cervantes to express my gratitude. 
But not before thanking Alexander Serebrenik and his group at the Technical University of Eindhoven for hosting me to develop this thesis in an international environment during 3 months.

La escritura y el desarrollo de una tesis doctoral es un viaje y, como tal, lo importante no es terminarla, sino el camino que se recorre y con quién lo compartes. 
He tenido la suerte de contar con el apoyo de numerosas personas, sin las cuales probablemente no estaría aquí y a las que quiero agradecérselo.

No podría empezar por otra persona que no fuera ella, sin duda quien más ha tenido que soportar mi frustración y ganas de dejarlo todo. 
No es fácil ver cómo tu entorno te dice que espera más de ti, que fueras a una gran empresa, consiguieras un buen trabajo y ganaras mucho dinero en una profesión que muchos envidian. 
Sin embargo, ella siempre me ha apoyado en mi meta de ser profesor e investigador porque sabía que era lo que realmente quería. 
Gracias por todo, Sandra.

Quería dar las gracias a mis padres y a mi familia en general por su apoyo incondicional y por haberme dado la oportunidad de estudiar lo que me gusta.

A mis amigos, que me han visto escribir la tesis en los lugares más inesperados y que han sido testigos de mi frustración y alegría.

A mis compañeros de clase, Pablo, Sergio y Enrique, sin duda personas brillantes (de las que siempre he esperado que se me pegue un poco) y con quienes tantas horas he pasado haciendo prácticas y desarrollándome como programador (a veces, en sitios tan inesperados como en un KFC).

A mis compañeros de CodeURJC, que tanto me han soportado y ayudado estos años.

A Jesús y Gregorio, dos prestigiosos investigadores de la casa que me han ayudado a un nivel casi equivalente al de mis tutores de tesis y me han ayudado notablemente a formarme como investigador.

Y por último (y no por ello menos importante), a mis tutores de tesis: Micael "Mica" Gallego y Francisco "Patxi" Gortázar, que han sido los que más han contribuido a que este viaje haya sido posible. Ellos me han formado no solo como investigador, sino como docente y como persona. No puedo estar más agradecido por su ayuda y (sobre todo) por su paciencia.

Gracias a todos.