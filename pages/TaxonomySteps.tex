The steps defined below to reproduce the taxonomy development process is based on the taxonomy development method developed by Nickerson et al.~\cite{Nickerson2013}.

\section{Set Up:}

\begin{enumerate}
	\item
	Create your own copy of taxonomy template (use
	\texttt{Taxononomy-Template.xslx}, published in the public repository of this work~\footnote{\url{https://github.com/Maes95/TFM-Buildability/tree/ReproducePackage/Taxonomy}})
	\item
	Download and run the Docker image with all results (take a bit time to
	download):
	\texttt{docker\ run\ -\/-rm\ -it\ -p\ 8888:8888}\\
		\hspace*{1cm} \texttt{-v\ \$PWD/analysis:/home/bugs/analysis}\\
		\hspace*{1cm} \texttt{maes95/project\_analysis:java-\textless{}java\_version\textgreater{}\ bash}
	\item
	Inside Docker container, run: \texttt{./runJupyterNotebook.sh}
	\item
	Open: \texttt{http://localhost:8888} and enter the token that you see
	in terminal to set up Jupyter Notebook.
	\item
	There are 3 levels of difficulty when you examine a trace:
	
	\begin{itemize}
		\item
		\textbf{Easy} -\textgreater{} You can get Dimensions/Characteristics
		just from trace
		\item
		\textbf{Medium} -\textgreater{} Insufficient trace information, you
		should goto project's notebook to inspect the full log. Navigate in
		Jupyter Notebook's page to \texttt{/bugs/analysis/\_notebooks/} and
		open the project. Go Kernel\textgreater Restart and Run All in menu.
		Find this line:
		\texttt{pa.view\_log\_by\_hash(errors,\textless{}hash\_id\textgreater{},\ 0)}
		and replace \texttt{\textless{}hash\_id\textgreater{}} with the
		corresponding hash in table to see the log (pressing enter in the
		cell).
		\item
		\textbf{Hard} -\textgreater{} Follow \emph{Medium} steps and you see
		an output like this:
		
		\texttt{Total\ commits:\ N\ \textbar{}\ Current\ commit:\ \textless{}hash\_id\textgreater{}\ \textbar{}\ Log:}
		
		Copy commit hash and go to Docker container:
		
		\begin{itemize}
			\item
			\texttt{cd\ bugs/projects/\textless{}project\textgreater{}}
			\item
			\texttt{git\ checkout\ -f\ \textless{}commit\_hash\textgreater{}}
			\item
			Build project:
			\texttt{../../configFiles/BuildFiles/build-\textless{}project\textgreater{}.sh}
			\item
			See output. Use \texttt{-\/-stacktrace} to get more info.
		\end{itemize}
	\end{itemize}
\end{enumerate}

\section{Notes}

\begin{itemize}
	\item
	You gona see in template 3 tabs, go to tab 3 \emph{Iteration 1}.
	\item
	There are 86 log traces divided into 5 groups, corresponding to the 5
	iterations.
	\item
	For each iteration it is possible to use a conceptual (use previous
	knowledge of the domain) or empirical (only use the data) approach.
	\item
	In base paper, they formally \textbf{define taxonomy as a set of
		dimensions}. Each dimension exiconsists of 2 or more characteristics:
	
	\begin{itemize}
		\item
		An object to be classified cannot have two different characteristics
		in one dimension (mutual exclusive restriction).
		\item
		Each object must have one of the characteristics of a dimension
		(collectively exhaustive restriction)
	\end{itemize}
	\item
	In paper, they \textbf{define attributes} of a good taxonomy:
	
	\begin{itemize}
		\item
		It's concise -\textgreater{} Not too many dimensions, the researcher
		must be able to have them in his head.
		\item
		It is robust -\textgreater{} It has a minimum number of dimensions
		that allow you to clearly separate objects.
		\item
		It is complete -\textgreater{} Classifies all objects within the
		domain Extendable -\textgreater{} Allows to include new dimensions
		when new objects appear
		\item
		It's explanatory -\textgreater{} Allows you to easily classify an
		object without having all the details of it.
	\end{itemize}
	\item
	In paper, they \textbf{define a series of End conditions}, which are
	the requirements to end a taxonomy. These can be objective (they are
	clearly defined in Table 2 of the paper) or subjective (the researcher
	estimates that the attributes cited above* have been satisfied).
	\item
	In paper, they \textbf{define the concept of meta-characteristic}, a
	highly understandable characteristic that gives rise to other
	characteristics. By way of example they propose the need to create a
	taxonomy of ``mobile applications'' and the meta-characteristics would
	be ``Interaction at high level between the user and the application'',
	where a dimension from this would be the communication, being the
	characteristics that the objects can take: informative, reporting or
	interactive. In our case, meta-characteristic is: \emph{Fail
		origin/cause}
\end{itemize}

\section{Steps}

\begin{enumerate}
	\item
	Take the a group N as a reference (17 traces)
	\item
	Select an approach:
	
	\begin{itemize}
		\item
		\textbf{Empirical}: Select a trace and try to extract a
		dimension and its characteristics. Add its
		dimensions/characteristics to the table and classify the trace. Go
		to next trace and try to classify with current
		dimensions/characteristics. If can't, create new
		dimensions/characteristics.
		\item
		\textbf{Conceptual}: Add conceptual dimensions/characteristics to the table
		and classify each trace based on them. If you can't classify a
		trace, create new dimensions/characteristics.
	\end{itemize}
	\item
	If you create new dimensions with new characteristics, update traces
	of Iteration N according to them.
	\item
	If all traces on Iteration N were classify, duplicate tab/sheet and
	call it \emph{Iteration N+1}. Then, go to \emph{Iteration N+1} and
	repeat steps 1-3.
\end{enumerate}