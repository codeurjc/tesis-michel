\section{Development tools}

\subsection{Docker} 

Docker~\footnote{\url{https://www.docker.com}} is an open source software platform to create, deploy and manage virtualized application containers on a common operating system. Docker uses resource isolation features of the Linux kernel (like cgroups and namespaces), to allow containers to run within a single instance of Linux, avoiding the overload of starting and maintaining virtual machines. We will use a docker to ensure that the experiment carried out can be reproduced in a simple way and obtain the same results.

\subsection{Anaconda}

Anaconda~\footnote{\url{https://www.anaconda.com/}} is a free and open distribution of Python and R languages, used in data science, machine learning and Big Data. We use this tool to manage the Python interpreter for the scripts that automate the tasks of the experiment inside a Docker container. In addition, it allows us to use Jupyter Notebook, an interactive tool for the browser that allows us to work easily with the data from the experiments.

\section{Design and Implementation}

In order to implement the process described in Chapter~\ref{sec:metodology} (Analysis process), the technologies presented above (Docker and Anaconda) have been used to build a Docker container with everything necessary to execute the experiment. 

The structure of the folders is as follows:

\begin{itemize}
	\item \textit{analysis/} Folder shared between the host and the container where the results are stored.
	\item \textit{configFiles/} Folder that includes the configuration of the execution of each project and its execution script.
	\item \textit{proyects/} Folder that includes the projects to be analyzed as Git repositories, navigable throughout their commits history.
	\item \textit{py/} Folder that includes Python scripts that provide the logic to execute all the project history builds and store the results.
\end{itemize} 

% \begin{figure}[h!]
% 	\begin{center}
% 		\includegraphics[width=16cm]{img/AppDiagram}
% 		\caption{Project structure}
% 		\label{fig:app}
% 	\end{center}
% \end{figure}

Figure~\ref{fig:app} shows the structure of the tool built to carry out the experiment.

Inside the container and in order to execute the \textbf{Check Build} phase, the following steps are followed:
\begin{enumerate}
	\item Python script $checkBuildHistory.py$ is launched, which will take the configuration file of a project together with its construction script.
	\item The construction of the project is executed from the source code provided in the commit N.
	\item The produced logs are stored, as well as the exit code of the construction process.
	\item Repeat steps 2 and 3 until N=0 (the first commit of the project) is reached.
\end{enumerate}

To approach the \textbf{Log Analysis} phase, we will use Jupyter Notebook to interact with the data in a simple way. This tool will take the logs and exit codes of each commit, selecting those that have failed to make a grouping of the logs. Figure~\ref{fig:jupyter} shows a fragment of these notebooks where the frequency of the errors found in the failed builds of the Spring Framework project is obtained. 

% \begin{figure}[h!]
% 	\begin{center}
% 		\includegraphics[width=14cm]{img/Jupyter}
% 		\caption{Screenshot of Jupyter Notebook - Log analysis for Spring Framework}
% 		\label{fig:jupyter}
% 	\end{center}
% \end{figure}


