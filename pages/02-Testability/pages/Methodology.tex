%The objective of our study is to analyze the testability of open-source projects throughout their commit history.
%To check if the tests can be executed, two previous steps are necessary: i) compiling the source code, and ii) compiling the test code. 
%If either of the previous steps fail, no test execution study can be performed.
%Projects from three different datasets have been selected to carry out the same study on each of them.

Let us define some terminology used through this chapter, the collection of projects that we analyze, and the methods used for the analysis.
%In this section we define the terminology used, the datasets studied, the methodology applied and the metrics to be collected.

\subsection{Definitions}
\label{subsec:definitions}
%We derive the terminology used in this paper from Sulir et al.'s work~\cite{Sulir:2016:QSJ:3001878.3001882}. According to it, the build process of projects programmed with compilable languages consists of following steps:

In this subsection, we extend the terminology defined in Section~\ref{sec:buildability:definitions} to distinguish between main source code (code that is used to compile and build the package) and testing code (code that verifies the correct behavior of the main source code, and will not be included in the package). 
We will also consider a further step consisting on running the tests, and collecting their results. 
The possible outcomes of a test execution are: \textbf{success}, if all test assertions are met, \textbf{failure}, if at least one of the assertions is not met, or \textbf{error}, if there is a problem in the test execution.
Therefore, considering the steps defined in Section~\ref{sec:buildability:definitions}, we would extend step 3 (\textit{execute the compiler to generate binary files from source code}) with 3 sub-steps:
\begin{inparaenum}[\bf(1)]
    \item \textit{compile the source code}
    \item \textit{compile the test code}, and
    \item \textit{execute the tests}.
\end{inparaenum}
With this in mind, we define:

%In particular, we define following concepts as follows:

\begin{itemize}
% \item \textbf{Commit}: a snapshot of the source code of a project, obtained by checking out a commit hash from git. Only commits on a given branch (usually \texttt{main} or \texttt{master}) will be considered, to keep history linear.
%    \item \textbf{Buildable source}: 
      %the ability of a version (commit) to be built successfully. We will consider the version as its source code and configuration files.
\item \textbf{Source-compilable}: a commit (snapshot) whose main source code can be successfully compiled. 
      % \item \textbf{Buildable test}: property of a version (commit) when its test code can be built successfully.
\item \textbf{Test-compilable}: a commit (snapshot) for which all tests can be successfully compiled (and has at least one test).
%\item \textbf{Testable Rate (for a given commit)}: ratio of success tests with respect to the total number of tests intended to run for that commit
\item \textbf{Fully Testable}: a commit (snapshot) for which all tests run with a success result.
      % property of a version (commit) when its tests can be executed with a successful result.

\end{itemize}

Based on this, we define the following metrics, that will characterize a project:
\begin{itemize}
    \item \textbf{Source Compilability}: ratio of \textit{source-compilable commits} with respect to the total number of commits.
    \item \textbf{Test Compilability}: ratio of \textit{test-compilable commits} with respect to the total number of commits.
    \item \textbf{Fully Testability}: ratio of \textit{fully testable commits} with respect to the total number of commits.
    % \item \textbf{Testability Rate\textsubscript{A}}: median of \textit{testable rate} for all commits in the project.
    % \item \textbf{Testability Rate\textsubscript{B}}: median of \textit{testable rate} for \textit{source-compilable commits} in the project.
    % \item \textbf{Testability Rate\textsubscript{T}}: median of \textit{testable rate} for \textit{test-compilable commits} in the project.
    % \item \textbf{Fully Testability\textsubscript{A}}: ratio of \textit{fully testable commits} with respect to the total number of commits.
    % \item \textbf{Fully Testability\textsubscript{B}}: ratio of \textit{fully testable commits} with respect to the number of \textit{source-compilable commits} of the project.
    % \item \textbf{Fully Testability\textsubscript{T}}: ratio of \textit{fully testable commits} with respect to the number of \textit{test-compilable commits} of the project.
\end{itemize}

% It should be noted that Sulir et al.~\cite{Sulir:2016:QSJ:3001878.3001882} use the term \textit{compilability} as the metric that measures how often the last commit of a project can be source-compilable. In this paper, we will distinguish between \textit{Source Compilability} and \textit{Test Compilability}. We define \textbf{Source Compilability} as a metric with respect to the total number of commits of the project metric which measures how often these commits can be source-compilable.
%Comparing with~\cite{Sulir:2016:QSJ:3001878.3001882}, it uses \textit{compilability} as the metric that measures how often the last commit of a project can be source-compilable. In this paper, we will distinguish between \textit{Source Compilability} and \textit{Test Compilability}. We define \textbf{Source Compilability} as a metric with respect to the total number of commits of the project metric which measures how often these commits can be source-compilable.
%On the other hand, we define \textbf{Test Compilability} as a metric with respect to the total number of source-compilable commits of the project that measures how often these commits can be test-compilable.

Each of these metrics tries to capture a \textit{ratio of progress} when we advance in the process towards running the tests. \textit{Source Compilability} informs on the ratio of snapshots that passed the first blocker: it only makes sense to test a snapshot if its main source could be built. This metric is quite similar to the one provided in the previous chapter and by Tufano et al.~\cite{tufano2017there}, when studying compilability. 
From this point on, we introduce \textit{Test Compilability}, to determine the ratio that pass the next blocker: tests can be built.
Thus, we define \textit{Testability} at the project level (as we did with \textit{compilability}), different from others who define it at the class or module level~\cite{bruntink2006empirical}.
% \patxi{¿Por qué este párrafo está separado del anterior? Va junto con Test Compiulability o con Fully Testability?}
Processing the result of \textit{all} tests in binary form gives us information on whether a change in that commit would be supported by the tests. 
This way of characterizing a commit as \textit{Fully Testable} or not is also used in Continuous Integration (CI) systems to validate if a change (commit) can be integrated with the rest of the code of a project.
We define \textbf{Fully Testability} as a metric with respect to the total number of commits of the project metric which measures how often these commits can be fully testable.
% In this paper, we define \textbf{Fully Testability} as a metric with respect to the total number of commits of the project metric which measures how often these commits can be fully testable.

% As for the Testability Rate, we compute several ratios of testable commits, which provide different information about the \textbf{Fully Testability} of a project. 
% \textbf{Fully Testability\textsubscript{A}} gives the raw fraction of all commits that are testable, useful to determine if there are many snapshots where the \textit{all} tests pass. 
% \textbf{Fully Testability\textsubscript{B}} is the fraction of source-compilable commits where the \textit{all} tests pass, and \textbf{Fully Testability\textsubscript{T}} is the fraction of test-compilable commits where the \textit{all} tests pass. 
% Both provide information about how likely it is that tests pass if the main source code, or the tests, could be built.

%We have defined as \textit{"Golden Rule"} what a developer would consider a \textit{healthy} project: \textit{the source code must compile, the test code must compile and the tests must run successfully}.Tests are a fundamental part of a software project, they are necessary to check its proper performance. In previous versions of a project, they are especially useful to check how the project behaved. Under this definition, \textbf{Fully Testability\textsubscript{A}} is proposed, considering all the commits of a project.

%The results of the work of Tufano et al. show that one of the pre-requisites (source code compilability) is quite likely not to be fulfilled for a considerable number of commits. This has led us to propose \textbf{Fully Testability\textsubscript{B}}, which considers only those commits in which the source code can be built.

%Like source compilability, test compilability can prevent the completion of all the steps up to the execution of the test. Thus, we propose the \textbf{Fully Testability\textsubscript{T}}, as it is interesting to know how about the tests in those commits where it is possible to build \textit{both} the source code and the test code.

% We understand testability as a metric that conveys information about how often commits of a project are testable. We have considered the testabilitycwith respect to the total number of commits of the project~\textbf{Fully Testability\textsubscript{A}}, with respect to the source-compilable commits~\textbf{Fully Testability\textsubscript{B}}, or with respect to the test-compilable commits~\textbf{Fully Testability\textsubscript{T}}.

\subsection{Dataset}

%The selected dataset projects have been obtained from different sources, but all of them are Java projects whose build system is Maven. Maven is recognized as the most widely used system for project building and test execution in Java, and has been declared as a standard for Java projects \cn. 

We decided, mainly for practical reasons, to work with Java projects that are built with Maven. 
Even when this focus our study on a fraction of all software, still Java is a very popular language, with many mature projects used in production. 
Maven is recognized as the most widely used system for project building and test execution in Java, and as we have seen in the previous chapter, projects that use this technology are more likely to have a high Source Compilability.

However, we did not consider all Java-Maven projects. 
Android applications written in Java require dependencies on the Android~SDK, and have some peculiarities that make those projects different from \textit{regular} Maven projects. 
So, we discarded Android projects, following~\cite{Sulir:2016:QSJ:3001878.3001882} in this respect. 
We also discarded projects for which tests lasted too much to run until completion: we set a per-project limit of 60 real-time days for running tests for all snapshots.

%In addition, Android projects have been discarded due to the dependencies on the Android SDK. The building and execution of tests are especially delicate when it comes to Android projects and are very different from the experiment we want to carry out. Other authors~\cite{Sulir:2016:QSJ:3001878.3001882} also consider leaving out such projects when mining and building Java projects.

To carry out the experiment, we selected a subset of projects from the ManySStuBs4J dataset~\cite{karampatsis2020often}, a well-known set of 100 Java Maven projects used for Program Repair research.
The ManySStuBs4J dataset does not include the git repositories, only their references to GitHub. 
As we have seen in the previous chapter, this results in some projects being unrecoverable if they have been deleted, made private or migrated to another git repository.
From this dataset we discarded two projects that are no longer available in the public repository and 11 Android projects. 
In addition, one project has not been included as it has not finished the execution of its experiment, which has taken longer than any other project.
Therefore, 86 projects have been selected.

\subsection{Methods}
\label{subsec:methods}

For each of the 86 projects we proceed as follows. First, we clone the git repository of the project (from GitHub\patxi{¿Los repos no están en el dataset? Aclarar esto porque hablamos del dataset pero luego se clonan de github}\michel{Aclarado en la sección anterior}), and we perform the following steps for each commit in its \textit{master/main} branch (the number of commits analyzed in each step can be seen in Figure~\ref{fig:methodology-process}):

\begin{enumerate}
    \item Checkout the commit.
    \item Compile the main source code by executing \textit{mvn clean compile -X}. If the command fails (the source compilation fails) the process stops. Otherwise, the commit is tagged as \textit{source-compilable}.
    %In any case, the log of the command execution is saved.
    \item Compile the test code by executing \textit{mvn test-compile}. If the command fails (the test compilation fails), or we detect there is no test in the snapshot, the process stops. Otherwise, the commit is tagged as \textit{test-compilable}.
    %In any case, the log of the command execution is saved.
    \item Run the project tests by executing \textit{mvn test}, checking if all tests passed (labeling the commit as \textit{fully testable}).
    In addition, we save the individual reports for each test class in XML format.
    \patxi{No sé si deberíamos indicar la info que guardas. Porque luego, para construir las otras métricas que van surgiendo hacen falta más datos que sólo si todos los tests han pasado. Problema, te pueden pedir (en el paper, no en la tesis) que en base a esos datos hagas más cosas.}
    \michel{Aclaro que guardamos los reportes de JUnit. De cara al articulo puedo dar más detalles del formato}
      %Otherwise, it means that at least one test has failed or has had some kind of error. 
    %In any case, we will save the results of each test as Maven Surefire Reports~\footnote{https://maven.apache.org/surefire/maven-surefire-plugin/} in XML and TXT for each test case.
\end{enumerate}

%There are commits in some projects that do not have tests, and where executing the mentioned build and test execution commands does not result in a failure. Thus, only those commits that have at least one test will be considered as compilable/testable.

\begin{figure}[h!]
    \centering    
    \begin{tikzpicture}[node distance=1cm,every node/.style={fill=white, font=\sffamily}, align=center]
    \footnotesize
    % Specification of nodes (position, etc.)
    \node (total)             [large]                                         { 407,579 Commits};
    \node (buildable)         [large, below of=total]                         { 103,097 Source-compilables commmits};
    \node (testBuildable)     [large, below of=buildable]                     { 93,925 Test-compilables commits};
    \node (success)           [large, below of=testBuildable]                 { 40,540 Testable commits};

    \draw[-Latex]             (total) -- (buildable);
    \draw[-Latex]             (buildable) -- (testBuildable);
    \draw[-Latex]             (testBuildable) -- (success);

\end{tikzpicture}
    \caption{Study overview \patxi{over all projects?}\michel{To discuss: Es de los 66 proyectos, después del filtrado de los que no tienen tests, no tengo claro como adelantar estos resultados, quizás haya que mover esta gráfica más adelante.}}
    \label{fig:methodology-process}
\end{figure}

%\subsection{Resulting metrics}
%\label{sec:metrics}

For each project, the following metrics are generated:

\begin{itemize}
    \item \textbf{Age:} Number of years since the first commit
    \item \textbf{LoC:} Number of lines of code of the project in its last commit
    \item \textbf{Total Commits:} Number of commits in the master or main branch
    \item \textbf{Source Buildable commits:} Number of commits that were source-compilable
    \item \textbf{Test-compilable commits:} Number of commits that were test-compilable
    \item \textbf{Testable commits:} Number of testable commits
    %\item \textbf{\# commits w/ failed tests:} Number of commits where at least one of the tests failed without any error occurring.
    %\item \textbf{\# commits w/ errors tests:} Number of commits where at least one of the tests produced an error. \michel{Añadir las definiciones de eror y failed test} \mica{¿Omitir estas dos últimas métricas?}
\end{itemize}


As described in~\ref{subsec:definitions}, based on the last four of them we also generate for each project the metrics already defined at the beginning of this section: \textbf{Source Compilability}, \textbf{Test Compilability} and \textbf{Fully Testability}.
We include as well metrics such as project age, lines of code (LoC) and total commits because these are widely used when analyzing software projects~\cite{yamashita2015revisiting,mannan2016understanding,rosen2015commit}, and can bring an idea of the size of the projects. 

% \textbf{Testability Rate\textsubscript{A}}, \textbf{Testability Rate\textsubscript{B}}, \textbf{Testability Rate\textsubscript{T}},
% \textbf{Fully Testability\textsubscript{A}}, \textbf{Fully Testability\textsubscript{B}}, \textbf{Fully Testability\textsubscript{T}}.
%     \item \textbf{Source Compilability:} ratio of \textit{Buildable Commits} over \textit{Total commits}.
%     \item \textbf{Test Compilability:} ratio of \textit{Test-compilable Commits} over \textit{Buildable Commits}.
%     \item \textbf{Fully Testability:} We consider three forms of testability, depending on the value on which it is calculated
%     \begin{itemize}
%         \item \textbf{Fully Testability\textsubscript{A}} percentage of \textit{Testable commits} over \textit{Total commits}.
%         \item \textbf{Fully Testability\textsubscript{B}} percentage of \textit{Testable commits} over \textit{Buildable Commits}.
%         \item \textbf{Fully Testability\textsubscript{T}} percentage of \textit{Testable commits} over \textit{Test-compilable Commits}.
%     \end{itemize}
% \end{itemize}

