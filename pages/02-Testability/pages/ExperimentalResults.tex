Once we performed our case study and determined the metrics that might better represent the projects, in this section we resort to describe the results in detail using the metrics defined in the previous section.

After running the experiment, we detected 20 projects in which not a single test was run on any commit in the history. 
The learning-spark project does not contain any tests in any commit of its history. 
The Mycat-Server project has a dependency on software that must be installed on the machine for it to be compiled (therefore its Source Compilability is 0\%). 
The Clojure project is a programming language and as such, its tests require a different execution method. 
The remaining 17 projects correspond to projects that contain multiple Maven modules and cannot be built and tested with the proposed methodology. 
\patxi{Ojo con esto que nos pueden decir que lo arreglemos, que un proyecto maven multimodulo no es nada raro}
\michel{Tenemos algunos proyectos multi-modulo que sí se pueden construir y testear con la metodología propuesta.}
%Despite the interest and the percentage of the total that these projects represent, we consider that they should be studied in a separate study. \patxi{Yo quitaba esta frase.}
Since we cannot execute any of the tests on any of these projects, we have decided to leave them out of our study, and from this point on we will only consider the remaining 66 projects.

Table~\ref{table:results-1} shows a summary of the experiment for the 66 projects. 
In the Count column we can see the magnitude of the study for the metrics defined in the methodology. 
The Mean ({\large$\mean{x}$}) and Median ({\large$\median{x}$}) columns show the trend at the project level.
We run a normality test for each metric in this table, finding that with the exception of \textit{Age}, none of the metrics shows a normal distribution. 
In general, differences between mean and median show large internal variability in each of the samples.

\begin{table}[!htb]
    \centering
    \caption{Absolute results.}
    \label{table:results-1}
    \begin{tabular}{|r|r|r|r|}
        \hline
        \multicolumn{1}{|c|}{\textbf{Metric}}& \multicolumn{1}{c|}{\textbf{Count}} & \multicolumn{1}{c|}{\textbf{\large{$\mean{x}$}}} & \multicolumn{1}{c|}{\textbf{\large{$\median{x}$}}} \\ \hline
        \textbf{Age}                      & 634.66                              & 9.62                               & 9.69                                 \\ \hline
        \textbf{LoC}                      & 11,143,058                         & 174,110.28                          & 26,973.50                             \\ \hline
        \textbf{Commits}                  & 407,579                           & 6,175.44                            & 2,831.00                              \\ \hline
        \textbf{Source-compilable commits} & 103,097                           & 1,562.08                            & 1,020.00                              \\ \hline
        \textbf{Test-compilable commit}    & 93,925                            & 1,423.11                            & 692.50                               \\ \hline
        \textbf{Fully Testable commits}   & 40,540                            & 614.24                             & 218.00                               \\ \hline
    \end{tabular}
\end{table}

Table~\ref{table:results-3} shows information on the buildability and testability metrics of the projects of the dataset. 
\patxi{Esta línea huérfana aquí se queda muy pobre. Habría que explicarlo un poco más.}
\michel{El problema aquí es que estos resultados se discuten en la siguiente sección. ¿Deberías fusionar ambas secciones?}

\begin{table}[h]
    \centering
    \caption{Mean ($\mean{x}$) and Median ($\median{x}$) values for Compilability and Testability of the projects.}
        \label{table:results-3}
        \begin{tabular}{|r|r|r|}
            \hline
            \multicolumn{1}{|c|}{\textbf{Metric}} & \multicolumn{1}{c|}{\textbf{Mean}} & \multicolumn{1}{c|}{\textbf{Median}} \\ \hline
            \textbf{Source Compilability}                         & 47.29\%                              & 47.12\%                                \\ \hline
            \textbf{Test Compilability\textsubscript{A}}          & 41.73\%                              & 39.22\%                                \\ \hline
            \textbf{Test Compilability\textsubscript{S}}          & 88.26\%                              & 97.35\%                                \\ \hline
            \textbf{Testability Rate\textsubscript{A}}           & 38.63\%                              & 34.88\%                                \\ \hline
            \textbf{Testability Rate\textsubscript{T}}           & 94.14\%                              & 99.53\%                                \\ \hline
            \textbf{Fully Testability\textsubscript{A}}          & 22.12\%                              & 14.88\%                                \\ \hline
            \textbf{Fully Testability\textsubscript{T}}          & 52.53\%                              & 59.32\%                                \\ \hline
    \end{tabular}
\end{table}


To illustrate the diversity of the results, we have divided the 66 projects into three groups of the same size according to their number of commits: large, medium and small. 
Figures~\ref{fig:many4j-1-bar-chart} (large projects), \ref{fig:many4j-2-bar-chart} (medium projects) and \ref{fig:many4j-3-bar-chart} (short projects) show the results for each project for each of the integer metrics as overlapping bars.
\patxi{Primero di que los agrupas por número de commits, y luego explicas lo de las barras. SI no queda raro}
\michel{Hecho}
% To illustrate the diversity of results across the projects, we have represented the different categories of commits per project in overlapped bars in Figures~\ref{fig:many4j-1-bar-chart} (large projects), \ref{fig:many4j-2-bar-chart} (medium projects) and \ref{fig:many4j-3-bar-chart} (short projects).
% The division of projects into the three categories (long, medium and short) is based on the number of commits.
It is easy to appreciate, just by checking colors, how each project tells a very different story. 

\begin{figure}[h!]
    \centering    
    \includegraphics[width=0.8\textwidth]{pages/02-Testability/images/Many4j 1-22.pdf}
    \caption{Project metrics (Large projects)}
    \label{fig:many4j-1-bar-chart}
\end{figure}%
\begin{figure}[h!]
    \centering    
    \includegraphics[width=0.8\textwidth]{pages/02-Testability/images/Many4j 23-44.pdf}
    \caption{Project metrics (Medium projects)}
    \label{fig:many4j-2-bar-chart}
\end{figure} 
\begin{figure}[h!]
    \centering    
    \includegraphics[width=0.8\textwidth]{pages/02-Testability/images/Many4j 45-66.pdf}
    \caption{Project metrics (Short projects)}
    \label{fig:many4j-3-bar-chart}
\end{figure}