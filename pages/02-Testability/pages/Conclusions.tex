In this chapter we have started a path to analyze to which extent past snapshots of a project can be tested. 
For that, we have conducted an empirical analysis of many Java projects from a well-known dataset. 
We also propose a framework for conducting further analysis, based on the different steps needed to successfully run tests for each snapshot. 
Using this framework, we have found that for more than half the snapshots, all tests cannot be run successfully.
However, the main result is the high variability of testability from project to project, even within relatively homogeneous dataset. 
% In this respect, we also discard some hypothesis on the influence of characteristics of projects (lines of code, number of commits, age) in testability.

%In this paper we offer a study of the testability of the history of 111 Java projects, which extends previous studies on the buildability of project history. 
%We also provide a complete study that discards some of the most common metrics of a project (lines of code, number of commits or its age) as factors that determine its testability. 
%We have found that in most projects it is not possible to reproduce the tests in the past.
%In addition, we conducted a preliminary study of the causes that can lead to low testability in Java projects.
%Analysing the projects we have found great variability, even among projects selected under the same criteria. 

We note that many projects cannot rely on running tests in past commits, as these won’t run or even compile.
Testing snapshots of the past is fundamental for the maintainability of old versions of the project which are still in production. 
Therefore, we expect more research in this area in the future. 
Fortunately, we have found some signals showing that good practices can be identified to increment testability of current snapshots (that will become past snapshots with time). 
We have also suggested some ways of increasing testability of past snapshots improving the methods we are using for building and running tests in them. 
Of course, extending our study to other samples of Java code, and to other programming languages, will improve our knowledge in this area.

%and also that some techniques could be used to of those snapshots, 
%Replication testing is essential to improve the maintainability of projects that need to be maintained in previous versions.
%Reproducing this study using other languages such as Python, Javascript or C++ could lead to interesting future work.