In order to gain a deeper understanding of the testability of the commit history of software projects, we will conduct a case study, which is aimed to be an exploratory research with quantitative data analysis. 
In this case study, we will make an exploratory analysis of different projects, checking if the proposed testability metric (\textit{Fully Testability}) is able to represent the reality of the testability of the projects in all their history.
To quantitatively analyze the projects, we will use a graph per project that represents the results per commit of the steps described in the methodology. 

As can be seen in Figure~\ref{fig:projects-1}, each sub-figure represents the commit history of a project, where the X axis shows the commit number (0 for the first commit of the project, 1 for the second, etc.). On the Y axis we observe a stacked bar for each commit, with the height being the number of tests executed in that commit (denoted on the left Y-axis) and the colors (red, orange and green) indicating the results of each test (failure, error and success respectively). Additionally, the graph includes for each commit on the X-axis the number of lines of code (in blue) and test code (in purple) for that commit, with the values denoted on the right Y-axis.

%%%%%%%%%%%%%%%%%%%%%%%%%%%%%%%%%%%%%%%%%%%%%%%%%%%%%%%%%%%%%%%%%%%%%
%                            FIRST SET                              %
%%%%%%%%%%%%%%%%%%%%%%%%%%%%%%%%%%%%%%%%%%%%%%%%%%%%%%%%%%%%%%%%%%%%%

The first set of projects that we are going to study (Figure~\ref{fig:projects-1}) have been selected as a representative set of projects that, taking a look at their visual representation, it seems that it is possible to run the tests in the past in almost all their history.
Attending to the Source (Src) Compilability, Test Compilability and Fully Testability metrics of these projects (shown in Table~\ref{table:projects-1}), we observe that the results are in line with the charts.
For instance, in the sub-figure of the \textit{DiskLruCache} project there are commits with failures or errors in their tests, which affects to some extent its Fully Testability. 
Specifically, this metric drops to 70.58\%, which means that in almost 30\% of the commits, at least one test results in a failure or error.

\begin{figure}[!htb]
    \centering
    \begin{minipage}{.5\textwidth}
        \centering
        \includegraphics[width=\textwidth]{pages/02-Testability/images/projects/jsoup.png}
        \label{fig:jsoup}
    \end{minipage}%
    \begin{minipage}{.5\textwidth}
        \centering
        \includegraphics[width=\textwidth]{pages/02-Testability/images/projects/spark.png}
        \label{fig:spark}
    \end{minipage}
    \begin{minipage}{.5\textwidth}
        \centering
        \includegraphics[width=\textwidth]{pages/02-Testability/images/projects/disklrucache.png}
        \label{fig:disklrucache}
    \end{minipage}%
    \caption{Set of projects 1: DiskLRUCache, Spark and JSoup}
    \label{fig:projects-1}
\end{figure}

\begin{table}[h!]
    \centering
    \resizebox{\textwidth}{!}{%
        \begin{tabular}{|r|r|r|r|}
        \hline
        \textbf{Project} & \textbf{Src Compilability}  & \textbf{Test Compilability} & \textbf{Fully Testability} \\ \hline
        jsoup            &  97.87\%                       & 97.29\%                      & 93.55\%                      \\ \hline
        spark            &  98.96\%                       & 92.84\%                      & 86.72\%                      \\ \hline
        DiskLruCache     &  100.00\%                      & 97.79\%                      & 70.58\%                      \\ \hline
        \end{tabular}
    }
    \caption{Metrics of set of projects 1: JSoup, Spark and DiskLRUCache}
    \label{table:projects-1}
\end{table}

%%%%%%%%%%%%%%%%%%%%%%%%%%%%%%%%%%%%%%%%%%%%%%%%%%%%%%%%%%%%%%%%%%%%%
%                           SECOND SET                              %
%%%%%%%%%%%%%%%%%%%%%%%%%%%%%%%%%%%%%%%%%%%%%%%%%%%%%%%%%%%%%%%%%%%%%

The second set of projects that we are going to study (Figure~\ref{fig:projects-2}) have been selected as a representative set of projects that, taking a look at their visual representation, it seems that it is possible to run the tests in the past in almost all their history, although in some of them we find multiple commits where failures or errors are located.
However, if we look again at the Fully Testability values (Table~\ref{table:projects-2}), we find that their values are surprisingly low. 
For example, the \textit{Checkstyle} project, despite the fact that a portion of its commits at the beginning of the project are not compilable, the figure seems to indicate that almost all of its tests pass. 
Its low Fully Testability value is due to a few tests failing consistently (in the order of 10 out of 1000) along the project history.
The other two projects expose a similar behavior, in the case of \textit{HikariCP} the failures are more visible in the graph (there are more tests per commit resulting in failures or errors) despite having the highest Fully Testability value of this set.

\begin{figure}[!htb]
    \centering
    \begin{minipage}{.5\linewidth}
        \centering
        \includegraphics[width=\textwidth]{pages/02-Testability/images/projects/fastjson.png}
        \label{fig:fastjson}
    \end{minipage}%
    \begin{minipage}{.5\linewidth}
        \centering
        \includegraphics[width=\textwidth]{pages/02-Testability/images/projects/checkstyle.png}
        \label{fig:checkstyle}
    \end{minipage}
    \begin{minipage}{.5\linewidth}
        \centering
        \includegraphics[width=\textwidth]{pages/02-Testability/images/projects/hikari.png}
        \label{fig:hikari}
    \end{minipage}%
    \caption{Set of projects 2: Fastjson, Checkstyle and HikariCP}
    \label{fig:projects-2}
\end{figure}

\begin{table}[h!]
    \centering
    \resizebox{\textwidth}{!}{%
        \begin{tabular}{|r|r|r|r|}
        \hline
        \textbf{Project} & \textbf{Src Compilability} & \textbf{Test Compilability} & \textbf{Fully Testability} \\ \hline
        HikariCP         &  96.02\%                      & 95.22\%                      & 16.92\%                      \\ \hline
        Checkstyle       &  77.72\%                      & 74.35\%                      & 0.78\%                      \\ \hline
        Fastjson         &  94.25\%                      & 88.35\%                      & 0.00\%                      \\ \hline
        \end{tabular}
        }
    \caption{Metrics of set of projects 1: Fastjson, Checkstyle and HikariCP}
    \label{table:projects-2}
\end{table}

Considering the above-mentioned results, Fully Testability is a very restrictive metric, as soon as a single test does not pass consistently in several commits, it drastically reduces the value of the metric. 
We can consider that we are missing information about the real testability of the project.
From the point of view of a practitioner or a researcher who needs to add a change to a past version and has to evaluate if the tests allow him to be sure not to introduce any regression in the code, knowing that in a commit 99.99\% of the tests pass may be enough.
This happened in the case of the \textit{Checkstyle} project, where there are commits with 3506 tests but only 1 fails.

Following this discussion, we propose a new and less restrictive way of measuring testability, which allows us to capture the information in a non-binary way. 
Instead of using a binary value (a commit is either Fully Testable or not) we will use a ratio. 
Hence, a commit has a \textbf{Testable Rate}, defined as the ratio of success tests with respect to the total number of tests run for that commit.
For a project, we would obtain the \textbf{Testability Rate} as the mean of the Testable Rate value of all commits in the history.

Table~\ref{table:projects-2-with-testability-rate} shows the Testability Rate results for the second set of projects.
This new metric reflects a higher testability value for this set of projects, which is in line with the reality of the tests executed.

\begin{table}[h!]
    \centering
    \resizebox{\textwidth}{!}{%
        \begin{tabular}{|r|r|r|r|r|}
        \hline
        \textbf{Project} & \multicolumn{1}{c|}{\textbf{\begin{tabular}[c]{@{}c@{}}Src \\ Compilability\end{tabular}}} & \multicolumn{1}{c|}{\textbf{\begin{tabular}[c]{@{}c@{}}Test \\ Compilability\end{tabular}}} & \multicolumn{1}{c|}{\textbf{\begin{tabular}[c]{@{}c@{}}Fully\\ Testability\end{tabular}}} & \multicolumn{1}{c|}{\textbf{\begin{tabular}[c]{@{}c@{}}Testability\\ Rate\end{tabular}}} \\ \hline
        HikariCP         &  96.02\%                      & 95.22\%                      & 16.92\%                      & 88.76\%                     \\ \hline
        checkstyle       &  77.72\%                      & 74.35\%                      & 0.78\%                       & 74.22\%                     \\ \hline
        fastjson         &  94.25\%                      & 88.35\%                      & 0.00\%                       & 87.80\%                     \\ \hline
        \end{tabular}
    }
    \caption{Metrics of set of projects 1: Fastjson, Checkstyle and HikariCP (including the new Testability Rate metric)}
    \label{table:projects-2-with-testability-rate}
\end{table}

%%%%%%%%%%%%%%%%%%%%%%%%%%%%%%%%%%%%%%%%%%%%%%%%%%%%%%%%%%%%%%%%%%%%%
%                            THIRD SET                              %
%%%%%%%%%%%%%%%%%%%%%%%%%%%%%%%%%%%%%%%%%%%%%%%%%%%%%%%%%%%%%%%%%%%%%

The third set of projects is shown in Figure~\ref{fig:projects-3}. 
In this case, we have chosen projects that appear to be very testable but only in a part of their history (due to problems in compiling the source code or test code).
Table~\ref{table:projects-3} shows the testability values for these projects. 

\begin{figure}[!htb]
    \centering
    \begin{minipage}{.5\linewidth}
        \centering
        \includegraphics[width=\textwidth]{pages/02-Testability/images/projects/zxing.png}
        \label{fig:zxing}
    \end{minipage}%
    \begin{minipage}{.5\linewidth}
        \centering
        \includegraphics[width=\textwidth]{pages/02-Testability/images/projects/okio.png}
        \label{fig:okio}
    \end{minipage}
    \begin{minipage}{.5\linewidth}
        \centering
        \includegraphics[width=\textwidth]{pages/02-Testability/images/projects/closure.png}
        \label{fig:closure-compiler}
    \end{minipage}%
    \caption{Set of projects 3: Zxing, Okio and Closure}
    \label{fig:projects-3}
\end{figure}

\begin{table}[h!]
    \centering
    \resizebox{\textwidth}{!}{%
        \begin{tabular}{|r|r|r|r|r|}
            \hline
            \textbf{Project} & \multicolumn{1}{c|}{\textbf{\begin{tabular}[c]{@{}c@{}}Src \\ Compilability\end{tabular}}} & \multicolumn{1}{c|}{\textbf{\begin{tabular}[c]{@{}c@{}}Test \\ Compilability\end{tabular}}} & \multicolumn{1}{c|}{\textbf{\begin{tabular}[c]{@{}c@{}}Fully\\ Testability\end{tabular}}} & \multicolumn{1}{c|}{\textbf{\begin{tabular}[c]{@{}c@{}}Testability\\ Rate\end{tabular}}} \\ \hline
            okio             & 36.65 & 36.06\%                      & 26.27\%                       & 36.01\%                     \\ \hline
            closure          & 60.18 & 59.84\%                      & 55.81\%                       & 59.84\%                     \\ \hline
            zxing            & 25.47 & 25.39\%                      & 24.80\%                       & 25.38\%                     \\ \hline
        \end{tabular}
    }
    \vspace*{0.5cm}
    \captionof{table}{Metrics of set of projects 3: Zxing, Okio and Closure}
    \label{table:projects-3}
\end{table}

These projects are characterized by a low Test Compilability. 
The Test Compilability metric is greatly affected by Source Compilability; if the code does not compile, the tests cannot be compiled. 
The reasons why source code cannot be compiled have been studied in previous chapter and in previous work~\cite{tufano2017there,Sulir:2016:QSJ:3001878.3001882}, and are caused mainly because of the impossibility to reproduce the context and retrieve the dependencies of a particular commit. 
For the okio, closure and zxing projects, the Source Compilability values are very close to Test Compilability values.
So, we can state that in these projects, if the source code compiles, in most cases the test code compiles as well. 
We can capture this idea in a new metric: \textbf{Test Compilability\textsubscript{S}}, the ratio of test-compilable commits with respect to the source-compilable commits.
The Test Compilability metric, when calculated considering all commits in the project history, is renamed to \textbf{Test Compilability\textsubscript{A}}.
Table~\ref{table:projects-3-B} shows the values of these two variants of Test Compilability, in addition to the Source Compilability values mentioned above.


\begin{table}[h!]
    \centering
    \resizebox{\textwidth}{!}{%
        \begin{tabular}{|r|r|r|r|r|r|}
        \hline
        \multicolumn{1}{|c|}{\textbf{Project}} & \multicolumn{1}{c|}{\textbf{\begin{tabular}[c]{@{}c@{}}Src \\ Compilability\end{tabular}}} & \multicolumn{1}{c|}{\textbf{\begin{tabular}[c]{@{}c@{}}Test \\ Compilability\textsubscript{A}\end{tabular}}} & \multicolumn{1}{c|}{\textbf{\begin{tabular}[c]{@{}c@{}}Test \\ Compilability\textsubscript{S}\end{tabular}}} & \multicolumn{1}{c|}{\textbf{\begin{tabular}[c]{@{}c@{}}Fully\\ Testability\end{tabular}}} & \multicolumn{1}{c|}{\textbf{\begin{tabular}[c]{@{}c@{}}Testability\\ Rate\end{tabular}}} \\ \hline
        okio                                   & 36.65\%                                                                                        & 36.07\%                                                                                         & 98.40\%                                                                                         & 26.28\%                                                                                     & 36.02\%                                         \\ \hline
        closure                                & 60.18\%                                                                                        & 59.84\%                                                                                         & 99.43\%                                                                                         & 55.82\%                                                                                     & 59.84\%                                         \\ \hline
        zxing                                  & 25.47\%                                                                                        & 25.39\%                                                                                         & 99.67\%                                                                                         & 24.80\%                                                                                     & 25.39\%                                         \\ \hline
        \end{tabular}
    }
    \captionof{table}{Extended metrics of set of projects 3: Zxing, Okio and Closure}
    \label{table:projects-3-B}
\end{table}

Following the idea of focusing on those commits where we can compile the test code, we should consider that it is possible that the commits that do not compile, never did and therefore their testability information is not useful to us.
% Once again we find that the testability metrics do not provide all the information. 
% When we cannot compile source code or test code, we lose testability information. 
For projects such as those in the third set, the testability of the commits where we can compile the tests gives us valuable information that neither the Fully Testability nor the Testability Rate capture, as both metrics are calculated from all commits in the history of the project.

To capture the testability of Test Compilable commits we define the following metrics:
\begin{itemize}
    \item \textbf{Testability Rate\textsubscript{T}}: mean of \textit{testable rate} for \textit{test-compilable commits} in the project.
    \item \textbf{Fully Testability\textsubscript{T}}: ratio of \textit{fully testable commits} with respect to the number of \textit{test-compilable commits} of the project.
\end{itemize}
The Fully Testability and Testability Rate metrics, when calculated considering \textit{all} the commits in the history, are renamed \textbf{Fully Testability\textsubscript{A}} and \textbf{Testability Rate\textsubscript{A}}.

In Table~\ref{table:projects-3-with-flavor-testability-rate} we include the new metrics defined for the third set of projects. 
We note that for all 3 projects, the new metrics more closely capture the testability of the commits we can compile.
The Fully Testability\textsubscript{T} shows us that for those commits where we can compile the tests, a high percentage can run all their tests with a success result. For this same set of commits, the Testability Rate\textsubscript{T} shows us values close to 100\%; almost all of the tests in these commits offer a success result.

\begin{table*}[h!]
    \centering
    \resizebox{\textwidth}{!}{%
        \begin{tabular}{|r|r|r|r|r|r|r|r|}
            \hline
            \rH{Project} & \rH{Src\\ Compilability} & \rH{Test\\ Compilability\textsubscript{A}} & \rH{Test\\ Compilability\textsubscript{S}} & \rH{Fully\\Testability\textsubscript{A}} & \rH{Fully\\Testability\textsubscript{T}} & \rH{Testability\\Rate\textsubscript{A}} & \rH{Testability\\Rate\textsubscript{T}} \\ \hline
            okio             & 36.65\%                        & 36.07\%                         & 98.40\%                         & 26.28\%                        & 72.85\%                        & 36.02\%                       & 99.86\%                       \\ \hline
            closure          & 60.18\%                        & 59.84\%                         & 99.43\%                         & 55.82\%                        & 93.27\%                        & 59.84\%                       & 99.99\%                       \\ \hline
            zxing            & 25.47\%                        & 25.39\%                         & 99.67\%                         & 24.80\%                        & 97.69\%                        & 25.39\%                       & 100.00\%                      \\ \hline
        \end{tabular}
    }
    \caption{Extended metrics of set of projects 3: Zxing, Okio and Closure (including the new Testability Rate\textsubscript{T} and Fully Testability\textsubscript{T} metrics)}
    \label{table:projects-3-with-flavor-testability-rate}
\end{table*}

\def \RQIII{On average, how many tests of a snapshot can be run with a `success' result?}

This case study helped us to define metrics that better describe a project's testability. 
These new metrics lead to a new research question: 

\textbf{RQ\textsubscript{2.3}}: ``\RQIII''
